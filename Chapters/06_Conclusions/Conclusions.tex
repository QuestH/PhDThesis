\chapter{Conclusions and Future Work}
\label{Conclusions}

\section{Overall Conclusions}

This thesis has presented investigations into the influence of considering the elastohydrodynamic (EHL) film implicitly at the roller-race conjunction in high-speed rolling element bearings modelling. A coupled tribological and dynamic approach was used to assess how the EHL film affects bearing dynamics at the operating speeds and loads of electrified powertrains.

This work reveals the impact of the lubricant film on bearing dynamics, particularly at high-speeds where the EHL film thickness increases substantially. The film thickness can exceed that of the contact deformation predicted by the dry Hertzian assumption, significantly influencing contact force and stiffness predictions which are underestimated in non-lubricated analyses. The following conclusions have been drawn from the results and analyses contained in this research:

\begin{enumerate}
	\item Experimental and tribological tests were conducted at speeds of up to 15~000~$rpm$, with a radial load of 750~$N$ applied to a shaft-bearing system. The results showed that dry modelling of a roller bearing underestimates the contact load by 149\% at 15~000~$rpm$. This discrepancy is attributed to the additional deformation incurred as a result of lubricant entrainment into the EHL contact. The central film thickness increased from 0.1~-~1.9~${\mu m}$ for a quasi-dynamic speed sweep from 0~-~15~000~rpm, leading to an increasing difference in contact load between dry and lubricated modelling with speed. However, at higher bearing loads, such as during system resonance, the contact load difference reduces to 21\% due to the contact deformation predicted by the dry Hertzian assumption closely matching the magnitude of the film (1.58~${\mu m}$ at 14~134~$rpm$). This therefore concludes that the dominating difference between these modelling techniques occurs at higher rotational speeds and low load. This matches the operating characteristics and magnitudes of electrified transmissions, and hence the consideration of the film at the roller-race conjunction cannot be neglected in their modelling.
		
	\item To assess the impact of this contact load calculation disparity on bearing dynamics, a coupled simulation approach was developed to integrate a lubricated bearing model within a system-level FMBD model. The approach implicitly considers the influence of the EHL film on the bearing contact mechanics at each time step of the dynamic analysis. Quasi-dynamic simulations were performed up to 21~000~$rpm$ using loading conditions from a first stage gear pair coupled to a 54~$kW$ permanent magnet synchronous motor (PMSM) with a peak torque of 68~$Nm$. Results show that the lubricant film increases contact stiffness by up to 24.9~\% at 21~000~$rpm$ when compared to conventional dry analyses. This is due to the non-linear force-deflection relationship at the contact. 
	The contribution of the lubricated contact stiffnesses increases the total bearing stiffness by 16.9~\%. This total stiffness consequently increases the natural frequency of the system from 3470~$Hz$ to 3542~$Hz$, corresponding to a 250~$rpm$ increase from the dry estimation at 12~500~$rpm$. Implicit inclusion of the film in the analysis was therefore shown to affect the predicted NVH response of the system. The increased bearing stiffness also reduced the total radial displacement of the shaft across the entire speed range.
	
	\item At high entrainment velocities of 30.7~$m \cdot \mathrm{s}^{-1}$, it was shown that the analytical film thickness equations deviate from the numerical solution and underestimate the estimated film thickness by 20.3\%. This corresponded to 21~000~$rpm$ in the bearings under investigation. An ANN was trained using a broad value range of tribological input data to predict EHL central film thickness. The training input data was constrained using the Greenwood regimes of lubrication, improving data cloud point density and achieving a coefficient of determination ($R^2$) of 0.99791 with just 600 data points.
	
	\item In static bearing tests, the central film thickness for entrainment velocities of 1.53~-~30.7~$m \cdot \mathrm{s}^{-1}$ were predicted by the ANN. Unlike the 20.3\% over-prediction of the central film using the analytical equations, the ANN under predicted the film by just 1.53\% at 21~000~$rpm$, having a mean squared error (MSE) of just 3.18~$\times 10^{-4}~\mu \mathrm{m}^2$ across the speed range. Static operating points with time-varying contact loads were also investigated at 10~000~$rpm$. Across the loading cycle, the ANN achieved an MSE of 3.89~$\times 10^{-6}~\mu \mathrm{m}^2$ when benchmarked against the numerical solution for a fluctuating contact load. This is a dramatic improvement over the MSE of the analytical model which was 1.24~$\times 10^{-1}~\mu\mathrm{m}^2$.
	
	\item The ANN was shown to be $\sim$1~500 times faster than the numerical solution with a very small margin of error when tested across 750 data points spanning a large value range of film thickness input variables. The ANN is a factor of $\sim$75 times slower than the analytical equation, but a factor of 3~$\times 10^{4}$ more accurate when comparing MSE performance of both against the numerical method.
	
	\item The ANN has been embedded within an FMBD system model to calculate EHL film thickness and consider it implicitly in the evaluation of the bearing dynamics. This establishes a workflow that will further enhance contact modelling in FMBD simulations.
\end{enumerate}


\section{Contributions to Knowledge} \label{Contribution to Knowledge}

The main novelties and contributions to knowledge from this thesis are summarised below:

\begin{enumerate}
	\item A novel experimental test rig was designed and constructed to measure the kinematic motion of a bearing at rotational speeds and loads up to 750~$N$ and 15~000~$rpm$ respectively. The bearing orbital motion was measured and used for conjunction and component level tribological analysis. The methodology of coupling experimental test with numerical tribological models has not been previously reported in this manner at these speeds. The outcome of this demonstrated the requirement of implicitly modelling the EHL film in future high-speed applications, since it contributed to a 149\% contact load increase at 15~000~$rpm$ when compared to conventional dry analyses.
	
	\item A coupled co-simulation approach was established to consider implicitly the EHL film in roller bearings within a high-speed system-level FMBD model. The model replicates the operating conditions of a 54~$kW$ permanent magnet synchronous motor (PMSM) coupled to a first stage gear pair, operating at speeds up to 21~000~$rpm$. This was the first time in open literature that an implicitly lubricated multi-physics bearing model has been considered in the context of electrified powertrain dynamics. The outcome of this demonstrated that the increased contact deflection, due to the lubricant film inclusion, increases total bearing stiffness by 24.9~\% at 21~000~rpm. This effectively behaves as a non-linear, speed dependant radial preload on the bearing. The contribution of this stiffness increased the natural frequency of the system, and hence affected NVH response. This proves the requirement to consider the EHL film implicitly for rolling element bearing modelling under the high speed, low load operating conditions of electrified powertrains.
	
	\item An ANN was trained using input data calculated using a 1D numerical EHL model. A wide value range of input variables necessary for the EHL central film thickness calculation were used to train the model. The value range was consistent with common machine element contacts, with its applicability also extending to gear pairs and cam contacts. The novel methodology of constraining the training data using the Greenwood regime ensured high data quality whilst only requiring 600 training data points. The ANN achieved an MSE of 3.89~$\times 10^{-6}~\mu \mathrm{m}^2$ when benchmarked against the numerical solution, whilst reducing calculation time by a factor of 1~500. This methodology will further contribute to the computational efficiency and accuracy of tribological ANNs.
	
	\item The ANN was embedded within an FMBD system-level model to calculate EHL film thickness and consider it implicitly in the evaluation of the bearing and system dynamics. The film thickness evaluation achieves the accuracy of numerical models without the associated computational limitations. This modelling method of combining component level ANN within a flexible system has not been previously reported.
\end{enumerate}

\section{Addressing the Research Questions}
The primary objective of this work was to investigate the interaction between tribology and dynamics in rolling element bearings, with particular focus on the significance of this multi-physics interaction in high-speed automotive applications. On reviewing the conclusions drawn from this thesis, it can be demonstrated that the research questions presented in Section \ref{Research Questions} have been addressed.

The findings of this thesis are to be implemented into the commercial software AVL EXCITE\textsuperscript{TM} M to support future rolling element bearing development. The influence of the lubricant film in dynamic analysis is of particular interest. The workflow of coupling contact level artificial neural networks into system level FMBD models is also of significance. This approach has applications beyond rolling element bearings, as a generic contact approach could also improve non-conformal contact modelling for gears and cams.

\section{Future Work}

The novel approaches presented in this thesis open up further opportunities to advance this area of research. These include:

\begin{itemize}
	\item For wider adoption of the ANN solution, the ability to extrapolate beyond the bounds of the training data must be addressed. Extrapolation errors in the film thickness estimation could yield unrealistic film thickness values, introducing potential impulses to the dynamic system. This could  affect the stability of an FMBD model and lead to divergence.
	\item Develop methodology for integrating ANNs into FMBD software. A suggested approach is to train the model using known input variables and kinematic conditions in a first stage static analysis. This trained model could then be called at each time step of the dynamic simulation. Alternatively, a library of trained models could be established using Greenwood regime constraints outlined in this work.
	\item Expand the training data set for the ANN to include experimental measurements from high-speed mini traction machines.
\end{itemize}