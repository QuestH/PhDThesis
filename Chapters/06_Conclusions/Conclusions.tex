\chapter{Conclusions and Future Works}
\label{Conclusions}

\section{Overall Conclusions}

This thesis presents a coupled tribological and dynamic modelling approach to investigate the influence of the EHL film on bearing dynamics under electrified vehicle operating conditions. The following conclusions have been drawn from the results and analyses contained in this research:

\begin{enumerate}
	\item This work highlights the significant impact of the lubricant film on bearing dynamics, particularly at high-speeds where the EHL film thickness increases substantially. The film thickness can exceed that of the contact deformation predicted by the dry Hertzian assumption, significantly influencing contact force and stiffness predictions which are underestimated in non-lubricated analyses.
	\item A coupled simulation approach has been developed, integrating an implicit lubricated bearing model within system-level, FMBD model. Simulations were performed with representative electrified powertrain conditions. Results show that the lubricant film increases contact stiffness by up to 24.9~\% at 21~000~rpm when compared to conventional dry analyses. This results in a 16.9~\% increase in total bearing stiffness, shifting the system’s natural frequency and affecting the predicted NVH response. This emphasises the requirement to include the EHL film in dynamic analysis of electrified powertrains.
	\item At high entrainment velocities, it is shown that the analytical film thickness equations deviate from the numerical solution and underestimate the estimated film thickness.
	\item An ANN has been trained to predict EHL film thickness based on numerical data, with the training input data constrained using the Greenwood regimes of lubrication for enhanced accuracy. This overcomes the inaccuracies of the regressed equations at high entrainment velocities, and predicts film thickness with marginal error.
	\item The ANN approach demonstrated exceptional computational efficiency, achieving approximately 1~500 times faster central film thickness calculations than traditional numerical methods.
	\item The ANN has been embedded within an FMBD system model to calculate EHL film thickness and consider it implicitly in the evaluation of the bearing dynamics. This establishes a workflow that will further enhance contact modelling in FMBD simulations.
\end{enumerate}


\section{Contributions to Knowledge} \label{Contribution to Knowledge}

The main novelties and contributions to knowledge from this thesis are summarised below:

\begin{enumerate}
	\item A novel methodology to measure bearing orbital motion for tribological analysis at high-speed has been presented. This enabled component and conjunction level tribodynamic analysis of a roller bearing under previously unreported speeds and loading conditions.
	\item A coupled co-simulation approach was established to integrate an implicit lubricated bearing model within a high-speed system-level FMBD model. The lubricated bearing model considers the EHL film in the evaluation of the bearing and system dynamics. This was the first time a multi-physics model of this nature has been reported in open literature.
	\item An ANN was trained using input data from the numerical EHL solution. This was used to predict bearing film thickness across a wide range of tribological input variables. The methodology of constraining the training data using the Greenwood regime has not been openly reported, and ensured high data quality and hence accurate predictions. This methodology will further contribute to the accuracy of tribological ANNs.
	\item The ANN was embedded within an FMBD system model to calculate EHL film thickness and consider it implicitly in the evaluation of the bearing and system dynamics. The film thickness evaluation achieves the accuracy of numerical models without the associated computational limitations. This modelling method of combining component level ANN within a flexible system has not been previously reported.
\end{enumerate}

\section{Addressing the Research Questions}
The primary objective of this work was to investigate the interaction between tribology and dynamics in rolling element bearings, with particular focus on the significance of this multi-physics interaction in high-speed automotive applications. On reviewing the conclusions drawn from this thesis, it can be demonstrated that the research questions presented in Section \ref{Research Questions} have been addressed.

The findings of this thesis are to be implemented into the commercial software AVL EXCITE\textsuperscript{TM} M to support future rolling element bearing development. The influence of the lubricant film in dynamic analysis is of particular interest. The workflow of coupling contact level artificial neural networks into system level FMBD models is also of significance. This approach has applications beyond rolling element bearings, as a generic contact approach could also improve non-conformal contact modelling for gears and cams.

\section{Future Works}

The novel approaches presented in this thesis open up further opportunities to advance this area of research.

\begin{itemize}
	\item For wider adoption of the ANN solution, the ability to extrapolate beyond the bounds of the training data must be addressed. Extrapolation errors in film thickness estimation could affect the stability of an FMBD model and lead to divergence.
	\item Develop methodology for integrating ANNs into FMBD software. A suggested approach is to train the model using known input variables and kinematic conditions in a first stage static analysis. This trained model could then be called at each time step of the dynamic simulation. Alternatively, a library of trained models could be established using Greenwood regime constraints outlined in this work.
	\item Expand the training data set for the ANN to include experimental measurements from high-speed mini traction machines. This would provide input data validated to a much wider range of operating conditions.
	\item Investigate the use of Physics Informed Neural Networks (PINNs) to have a more data driven approach to tribological modelling. This would allow more complex calculations and reduce the risk of extrapolation error.
\end{itemize}