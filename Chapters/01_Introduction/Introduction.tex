\chapter{Introduction}
\label{Introduction}

The automotive industry is currently transitioning into the next phase of powertrain technology. As automotive manufacturers are forced to meet tightening fleet-wide emissions regulations, the electrified vehicle market share is increasing. The EU has established policy \cite{EUL110/5} that requires all new cars and vans sold in Europe to be zero-emission by 2035 as part of a broader strategy for a climate-neutral EU by 2050.

EU overwhelming market share of 80\% by 2030 

BEV PHEV 

The type and 

Recent EU stagnation in the market, due in part to uncertainty surrounding emission rules. - push toward hybrid.

nevertheless, changing industry and a move towards more complex powertrains, incorporating electrified components. High speed. 

Europe - ambitious climate policies and stringent emissions legislation. Competition with Chinese subsidised markets

China, electrified passenger car segment is domination 

Investment in battery technology is advancing the growth of 

Whilst the industry is susceptible geopolitical and market influences, one thing is clear - future powertrains will rely either in whole or in part on electrification.


The use of ultra-high speed and low load motors introduces new challenges regarding NVH (Noise, Vibration and Harshness) and the tribology of interacting conjunctions. The compact, lightweight and efficient motors operate under significantly different working conditions and are subject to different underlying physics; such as regime of lubrication, dynamic response and magneto-mechanical interactions. This style of powertrain architecture therefore involves high-speed bearing operation in both the motor and transmission.

With a trend towards cost saving zero-prototype development, the use of simulation tools in modern powertrain development is also growing. Significant cost reductions can be achieved using commercial flexible multi-body dynamic solvers to replicate system level operation of these vehicles. Multi-system vehicle powertrain concepts are pushing complexity of simulation models and this requires accurate and robust component level understanding. Associated performance characteristics of the bearings such as friction and wear, thermal stability and generated vibration and noise must be accurately modelled at the development stage to ensure full system success \cite{Wensing1972a}. 

\section{Numerical EHL Model}

\subsection{Numerical EHL Model}

Reynold’s equation \cite{Reynolds1886} is the governing equation of fluid film lubrication theory. For Newtonian fluids it can be derived from the full Navier-Stokes equations making the following assumptions, primarily the neglection of inertial forces and only retaining viscous forces on the lubricant \cite{Gohar1988}:
\begin{enumerate} % TODO REQUIRES DIRECTIONS ADDING
	\item Body forces are negligible (mass of film is negligible)
	\item Pressure is constant through the lubricant film (z-direction) due to thin film (dimensions of the region of pressure are typically 100 times the central film thickness).
	\item No slip at boundaries
	\item Lubricant flow is laminar (low Reynolds number)
	\item Inertia and surface tension forces are negligible compared with viscous forces (working fluid has low mass and low acceleration)
	\item Shear stress and velocity gradients are only significant across the lubricant film %ADD DIRECTION OF THIS!!!!
	\item The lubricant behaves as a Newtonian fluid
	\item Lubricant viscosity in constant across the film %ADD DIRECTION!!!
	\item The lubricant boundary surfaces are parallel or at a small angle with respect to each other
\end{enumerate}

Reynolds equation is a second order, non-linear partial differential equation. It is made up of the pressure induced terms (Poiseuille flow) and the boundary velocity-induced term (Couette flow). 

For the line contact problem, such as that at the conjunction between a cylindrical roller and race, dimensions in the side-leakage direction,$y$, are much bigger than the direction of entraining motion, $x$. Pressure in $y$ direction is assumed constant due to the negligible gradient, and the contact can be analysed in 1-dimension. The assumption is valid in the contact apart from small regions near the edge where the roller profile changes. A simplified 1-dimensional version of Reynolds equation can therefore be used:
\begin{equation}\label{eq1.1}
	\frac{\partial}{\partial x}\left[\frac{\rho h^{3}}{6 \eta}\left(\frac{\partial p}{\partial x}\right)-\rho h u\right]=2 \frac{\partial(\rho h)}{\partial t}
\end{equation}

To solve Reynolds equation numerically, it must first be discretized and then solved using the finite-difference method. The following procedure explains this discretization.

Due to the steady state nature of the investigations, with the absence of shock loading, the transient squeeze term can be removed:
\begin{equation}\label{eq1.2}
	\frac{\partial}{\partial x}\left[\frac{\rho h^{3}}{6 \eta}\left(\frac{\partial p}{\partial x}\right)-\rho h u\right]=0
\end{equation}

Due to the many orders of magnitude differences between lubricant film thickness (µm) and pressures (GPa), the numerical solution often becomes unstable. Dimensionless parameters are therefore defined to remove this instability. These are as follows:
\begin{equation}\label{eq1.3}
	\begin{aligned}
		U &=\frac{u}{u_{a v}} & \partial x &=a \partial X \\
		\mathrm{X} &=\frac{x}{\mathrm{a}} & \partial \rho &=\rho_{0} \partial \bar{\rho} \\
		\bar{\rho} &=\frac{\rho}{\rho_{0}} & \partial \eta &=\eta_{0} \partial \bar{\eta} \\
		\bar{\eta} &=\frac{\eta}{\eta_{0}} & \partial h &=\frac{a^{2}}{R_{z x}} \partial H \\
		\mathrm{H} &=\frac{h R_{x}}{a^{2}} & \partial p &=p_{h} \partial P \\
		\mathrm{P} &=\frac{p}{p_{h}} & \\
		\mathrm{~W}^{*} &=\frac{w}{E_{r} R_{z x} L} &
	\end{aligned}
\end{equation}

Terms in the simplified Reynolds equation are replaced with dimensionless parameters. Similar terms are then grouped and rearranged to give the final form:
\begin{equation}\label{eq1.4}
	\frac{\partial}{\partial X}\left[\frac{\bar{\rho} H^{3}}{6 \bar{\eta}}\left(\frac{\partial P}{\partial X}\right)\right]=\Psi\left[\frac{\partial}{\partial X} \bar{\rho} H U\right]
\end{equation}

where
\begin{equation}\label{eq1.5}
	\Psi=\frac{12 u_{a v} R_{z x}^{2} \eta_{0}}{p_{h}}
\end{equation}

Grouping terms for simplicity
\begin{equation}\label{eq1.6}
	M=\frac{\bar{\rho} H^{3}}{6 \bar{\eta}}
\end{equation}

\begin{equation}\label{eq1.7}
	Q=\bar{\rho} H
\end{equation}

Making substitutions
\begin{equation}\label{eq1.8}
	\frac{\partial}{\partial X}\left[M\left(\frac{\partial P}{\partial X}\right)\right]=\Psi \frac{\partial}{\partial X}[Q U
\end{equation}

\begin{equation}\label{eq1.9}
	\left[M \frac{\partial^{2} P}{\partial X^{2}}+\left(\frac{\partial M}{\partial X}\right) \frac{\partial P}{\partial X}\right]=\Psi\left[U \frac{\partial Q}{\partial X}+Q \frac{\partial U}{\partial X}\right]
\end{equation}

The final term is removed, as velocity, $U$, is independent of $x$ when no stretching of the surfaces occurs. This is then differentiated to give:
\begin{equation}\label{eq1.10}
	\frac{\partial M}{\partial X}=\frac{\partial}{\partial X}\left[\frac{\bar{\rho} H^{3}}{6 \bar{\eta}}\right]=\frac{H^{2}}{2 \bar{\eta}}\left[\left(\frac{H}{3}\right) \frac{\partial P}{\partial X}+\bar{\rho} \frac{\partial H}{\partial X}-\left(\frac{\bar{\rho} H}{2 \bar{\eta}}\right) \frac{\partial \bar{\eta}}{\partial X}\right]
\end{equation}
and
\begin{equation}\label{eq1.11}
	\frac{\partial Q}{\partial X}=\frac{\partial}{\partial X}[\bar{\rho} H]=H \frac{\partial \bar{\rho}}{\partial X}+\bar{\rho} \frac{\partial H}{\partial X}
\end{equation}

Substituting into Equation \ref{eq1.9} gives the following:
\begin{equation}\label{eq1.12}
	\frac{\bar{\rho} H^{3}}{6 \bar{\eta}} \frac{\partial^{2} P}{\partial X^{2}}+\frac{H^{2}}{2 \bar{\eta}}\left[\frac{H}{3} \frac{\partial \bar{\rho}}{\partial X}+\bar{\rho} \frac{\partial H}{\partial X}-\frac{\bar{\rho} H}{2 \bar{\eta}} \frac{\partial \bar{\eta}}{\partial X}\right] \frac{\partial P}{\partial X}-\Psi U\left[H \frac{\partial \bar{\rho}}{\partial X}+\bar{\rho} \frac{\partial H}{\partial X}\right]=0
\end{equation}

\begin{equation}\label{eq1.13}
	\frac{\partial^{2} P}{\partial X^{2}}+\frac{3}{\bar{\rho} H}\left[\frac{H}{3} \frac{\partial \bar{\rho}}{\partial X}+\bar{\rho} \frac{\partial H}{\partial X}-\frac{\bar{\rho} H}{2 \bar{\eta}} \frac{\partial \bar{\eta}}{\partial X}\right] \frac{\partial P}{\partial X}-\frac{6 \bar{\eta}}{\bar{\rho} H^{3}} \Psi U\left[H \frac{\partial \bar{\rho}}{\partial X}+\bar{\rho} \frac{\partial H}{\partial X}\right]=0
\end{equation}

The final form of the equation is therefore:
\begin{equation}\label{eq1.14}
	\frac{\partial^{2} P}{\partial X^{2}}+\left[\frac{1}{\bar{\rho}} \frac{\partial \bar{\rho}}{\partial X}+\frac{3}{H} \frac{\partial H}{\partial X}-\frac{3}{2 \bar{\eta}} \frac{\partial \bar{\eta}}{\partial X}\right] \frac{\partial P}{\partial X}-\frac{6 \bar{\eta}}{H^{2}}\left[\frac{1}{\bar{\rho}} \frac{\partial \bar{\rho}}{\partial X}+\frac{1}{H} \frac{\partial H}{\partial X}\right] \Psi U=0
\end{equation}

\subsection{Finite Difference Formulation}

For finite difference formulation, the central difference formula based on Taylor series expansion \cite{Hoffmann2000} is used. The second derivative of pressure using second order central discretization for the spatial domain is therefore:

\begin{equation}\label{eq1.15}
	\frac{\partial^2 P}{\partial X^2}=\frac{P_{i-1}-2 P_i+P_{i+1}}{\Delta X^2}
\end{equation}

and the first derivative is given by:

\begin{equation}\label{eq1.16}
	\frac{\partial P}{\partial X}=\frac{P_{i+1}-P_{i-1}}{2 \Delta X}
\end{equation}

Replacing terms in the final form of the discretized Reynolds equation:

\begin{equation}\label{eq1.17}
	\frac{P_{i-1}-2 P_i+P_{i+1}}{\Delta X^2}+\left[\frac{1}{\bar{\rho}} \frac{\partial \bar{\rho}}{\partial X}+\frac{3}{H} \frac{\partial H}{\partial X}-\frac{3}{2 \bar{\eta}} \frac{\partial \bar{\eta}}{\partial X}\right] \frac{P_{i+1}-P_{i-1}}{2 \Delta X}-\frac{6 \bar{\eta}}{H^2}\left[\frac{1}{\bar{\rho}} \frac{\partial \bar{\rho}}{\partial X}+\frac{1}{H} \frac{\partial H}{\partial X}\right] \Psi U=0
\end{equation}

\begin{equation}\label{eq1.18}
	\frac{P_{i-1}+P_{i+1}}{\Delta X^2}+\left[\frac{1}{\bar{\rho}} \frac{\partial \bar{\rho}}{\partial X}+\frac{3}{H} \frac{\partial H}{\partial X}-\frac{3}{2 \bar{\eta}} \frac{\partial \bar{\eta}}{\partial X}\right] \frac{P_{i+1}-P_{i-1}}{2 \Delta X}-\frac{6 \bar{\eta}}{H^2}\left[\frac{1}{\bar{\rho}} \frac{\partial \bar{\rho}}{\partial X}+\frac{1}{H} \frac{\partial H}{\partial X}\right] \Psi U=\frac{2 P_i}{\Delta X^2}
\end{equation}

Pressure at each node points can then be represented by:

\begin{equation}\label{eq1.19}
	P_i=\frac{\frac{P_{i-1}+P_{i+1}}{\Delta X^2}+\left[\frac{1}{\bar{\rho}} \frac{\partial \bar{\rho}}{\partial X}+\frac{3}{H} \frac{\partial H}{\partial X}-\frac{3}{2 \bar{\eta}} \frac{\partial \bar{\eta}}{\partial X}\right] \frac{P_{i+1}-P_{i-1}}{2 \Delta X}-\frac{6 \bar{\eta}}{H^2}\left[\frac{1}{\bar{\rho}} \frac{\partial \bar{\rho}}{\partial X}+\frac{1}{H} \frac{\partial H}{\partial X}\right] \Psi U}{2\left(\frac{1}{\Delta X^2}\right)}	
\end{equation}

Simplified to:

\begin{equation}\label{eq1.20}
	P_i=\frac{P_{x x}+P_x-E}{2\left(\frac{1}{\Delta X^2}\right)}
\end{equation}

where

\begin{equation}\label{eq1.21}
	P x x=\frac{P_{i-1}+P_{i+1}}{\Delta X^2}
\end{equation}

\begin{equation}\label{eq1.22}
	P x=\frac{P_{i+1}-P_{i-1}}{2 \Delta X}\left[\frac{1}{\bar{\rho}} \frac{\partial \bar{\rho}}{\partial X}+\frac{3}{H} \frac{\partial H}{\partial X}-\frac{3}{2 \bar{\eta}} \frac{\partial \bar{\eta}}{\partial X}\right]
\end{equation}

\begin{equation}\label{eq1.23}
	E=\frac{6 \bar{\eta}}{H^2}\left[\frac{1}{\bar{\rho}} \frac{\partial \bar{\rho}}{\partial X}+\frac{1}{H} \frac{\partial H}{\partial X}\right] \Psi U
\end{equation}