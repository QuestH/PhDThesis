\chapter{Introduction} \label{Introduction}

The automotive industry is currently transitioning into the next phase of powertrain technology. As automotive manufacturers are forced to meet tightening fleet-wide emissions regulations, the electrified vehicle market share is increasing. The EU has established policy \cite{EUL110/5} that requires all new cars and vans sold in Europe to be zero-emission by 2035 as part of a broader strategy for a climate-neutral EU by 2050.

EU overwhelming market share of 80\% by 2030 

BEV PHEV 

The type and 

Recent EU stagnation in the market, due in part to uncertainty surrounding emission rules. - push toward hybrid.

nevertheless, changing industry and a move towards more complex powertrains, incorporating electrified components. High speed. 

Europe - ambitious climate policies and stringent emissions legislation. Competition with Chinese subsidised markets

China, electrified passenger car segment is domination 

Investment in battery technology is advancing the growth of 

Whilst the industry is susceptible geopolitical and market influences, one thing is clear - future powertrains will rely either in whole or in part on electrification.


The use of ultra-high speed and low load motors introduces new challenges regarding NVH (Noise, Vibration and Harshness) and the tribology of interacting conjunctions. The compact, lightweight and efficient motors operate under significantly different working conditions and are subject to different underlying physics; such as regime of lubrication, dynamic response and magneto-mechanical interactions. This style of powertrain architecture therefore involves high-speed bearing operation in both the motor and transmission.

With a trend towards cost saving zero-prototype development, the use of simulation tools in modern powertrain development is also growing. Significant cost reductions can be achieved using commercial flexible multi-body dynamic solvers to replicate system level operation of these vehicles. Multi-system vehicle powertrain concepts are pushing complexity of simulation models and this requires accurate and robust component level understanding. Associated performance characteristics of the bearings such as friction and wear, thermal stability and generated vibration and noise must be accurately modelled at the development stage to ensure full system success \cite{Wensing1972a}. 

Simulating electrified powertrains using flexible multi-body dynamic (FMBD) models can enable substantial cost and time savings for automotive manufacturers due to a reduced need for physical prototyping. With increasing complexity and operational speeds of these systems, the accuracy at the component level is of major importance. Bearings are crucial structural components and their dynamic response significantly affects the behaviour of the interconnected structures.




\section{Research Questions} \label{Research Questions}

Fundamental aim of thesis

\begin{enumerate}
	\item
	\item
	\item
	\item
	\item
\end{enumerate}

\section{Contribution to Knowledge} \label{Contribution to Knowledge}

Novelties in thesis

\begin{enumerate}
	\item
	\item
	\item
	\item
	\item
\end{enumerate}


\section{Structure of Thesis} \label{Structure of Thesis}

The research in this thesis is presented with the following structure:

\paragraph{Chapter 1} introduces the topic of the thesis and provides the specific research questions that the work aims to answer.

\paragraph{Chapter 2} provides a review of the literature pertinent to the thesis topic, as well as covering the necessary aspects required to address the research questions.

\paragraph{Chapter 3} presents a high-speed experimental test rig which was used to obtain boundary conditions for tribological models. The governing equations for the thesis are also introduced in this chapter.

\paragraph{Chapter 4} integrates the tribological models into a flexible system level multi-body dynamic model. High-speed simulations are performed with operating conditions representative of those in an electrified transmission. The elastohydrodynamic film is modelled implicitly at the roller-race conjunction.

\paragraph{Chapter 5} addresses the computational efficiency of implicitly coupling the EHL film in dynamic simulation. An artificial neural network is trained using a wide design space of input variables required for the EHL film estimation. This replaces the analytical film thickness approach in Chapter 4 to overcome the shortcomings of the regressed equations.

\paragraph{Chapter 6} presents the general conclusions of the work, and outlines potential future work that could succeed this research.