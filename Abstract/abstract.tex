% ************************** Thesis Abstract *****************************
% Use `abstract' as an option in the document class to print only the titlepage and the abstract.

\begin{abstract}

Roller bearings are critical components in electrified vehicle powertrains. They are often performance limiting, introducing NVH (Noise, Vibration and Harshness), tribological and wear challenges. With a push towards achieving zero-prototype development, the use of advanced simulation tools to accurately predict their behaviour at both component and system level is becoming more prevalent.

Modern electrified motors and transmissions operate at considerably higher speeds than traditional internal combustion engine powertrains. This leads to much higher lubricant entrainment velocities at the roller-race conjunction in the bearings. Consequently, the elastohydrodynamic (EHL) film can be of the same order of magnitude and even exceed that of the contact deformation predicted by the dry Hertzian assumption. This significantly influences the contact mechanics and hence total bearing stiffness.

This research presents a coupled tribological and dynamic modelling approach for high-speed rolling element bearings. Experimental studies using a novel test rig highlight the influence of the EHL film, with measured bearing motion serving as boundary conditions for tribological analyses. A coupled simulation approach is developed, integrating an implicit lubricated bearing model within a high-speed, system level, flexible multi-body dynamic model. Results at both the contact and system level are evaluated to quantify the differences between lubricated and non-lubricated analyses. An artificial neural network (ANN) is trained to predict EHL film thickness across a wide range of bearing geometries, tribological parameters and operating conditions. The ANN is then employed within the coupled simulation as a more accurate alternative to the regressed film thickness equations, with its performance evaluated against numerical and analytical approaches in terms of predictive accuracy and computational efficiency.

This multi-physics approach improves the understanding of the interaction between tribological behaviour and system dynamics. The deeper understanding of these matters is expected to support more objective development of these modern powertrains in future to enhance their efficiency, durability and NVH refinement.
 
\paragraph{Keywords} High-speed; roller bearings; tribodynamics; elastohydrodynamics; EHL; flexible multi-body dynamics; artificial neural network; ANN

\end{abstract}
