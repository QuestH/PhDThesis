% ************************** Thesis Abstract *****************************
% Use `abstract' as an option in the document class to print only the titlepage and the abstract.
%\begin{abstract}
\addcontentsline{toc}{chapter}{Nomenclature}

\chapter*{Nomenclature}

The nomenclature of all symbols used in the body of this work is given in the following sections. This work combines concepts from various research domains, including multi-body dynamics, elastohydrodynamic lubrication and computer science. Since each field has its own established notation, overlap for certain symbol invariably occurs. The nomenclature is therefore divided by chapter to maintain convention and reduce potential confusion.

\paragraph{Chapter 3}

\begin{align*}
	&a && \text { Acceleration }\left(\mathrm{m \cdot s}^{-2}\right) \\
	&b && \text { Half-length of the contact (mm) } \\
	&C && \text { Radial clearance ( } \mu \mathrm{m} \text { ) } \\
	&D_r && \text { Diameter of roller (mm) } \\
	&D_p && \text { Pitch diameter (mm) } \\
	&E_r && \text { Equivalent (reduced) elastic modulus (Pa) } \\
	&F_x && \text { Radial load in x-direction (N) } \\
	&F_y && \text { Radial load in y-direction (N) } \\
	&f_{b p i} && \text { Ball pass frequency of inner race (Hz) } \\
	&f_{b p o} && \text { Ball pass frequency of outer race (Hz) } \\
	&f_{s h a f t} && \text { Shaft rotational frequency (Hz) } \\
	&G^* && \text { Dimensionless equivalent geometry (-) } \\
	&h_c && \text { Central film thickness (m) } \\
	&k && \text { Stiffness }\left(\mathrm{N \cdot m}^{-1}\right) \\
	&L && \text { Roller length (m) } \\
	&n && \text { Exponent of localized deflection (-) } \\
	&\mathrm{N} && \text { Number of rolling elements (-) } \\
	&p && \text { Contact pressure (Pa) } \\
	&r_{i n} && \text { Radius of inner race (m) } \\
	&R_{z x} && \text { Equivalent radius of contact (m) } \\
	&u && \text { Speed of entraining motion }\left(\mathrm{m} \cdot \mathrm{s}^{-1}\right) \\
	&U^* && \text { Dimensionless speed parameter }(-) \\
	&v && \text { Velocity }\left(\mathrm{m} \cdot \mathrm{s}^{-1}\right) \\
	&W && \text { Contact load (N) } \\
\end{align*}

\pagebreak

\begin{align*}
	&W^* && \text { Dimensionless load parameter }(-) \\
	&x && \text { Displacement in x-direction }(\mathrm{m}) \\
	&x_c && \text { Conjunction x-coordinate }(-) \\
	&y && \text { Displacement in y-direction }(\mathrm{m}) \\
	&y_c && \text { Conjunction y-coordinate }(-) \\
\end{align*}
	
\paragraph{Greek Symbols}
\begin{align*}
	&\theta && \text { Angular position (rad) } \\
	&\alpha && \text { Pressure viscosity coefficient }\left(GPa^{-1}\right) \\
	&\delta && \text { Contact deflection (m) } \\
	&\lambda && \text { Stribeck parameter (-) } \\
	&\eta_0 && \text { Atmospheric lubricant dynamic viscosity }\left(\mathrm{Pa} \cdot \mathrm{s}\right) \\
	&\eta && \text { Lubricant dynamic viscosity }\left(\mathrm{Pa} \cdot \mathrm{s}\right) \\
	&\rho && \text { Lubricant density }\left(\mathrm{kg} \cdot \mathrm{m}^{-3} \right) \\
	&\rho_0 && \text { Atmospheric lubricant density }\left(\mathrm{kg} \cdot \mathrm{m}^{-3} \right) \\
	&\sigma && \text { Composite surface roughness (m) } \\
	&\omega_c && \text { Angular velocity of cage }\left(\mathrm{rad \cdot s}^{-1}\right) \\
	&\omega_{r i} && \text { Angular velocity of inner race }\left(\mathrm{rad \cdot s}^{-1}\right) \\
	&\omega_s && \text { Angular velocity of shaft }\left(\mathrm{rad \cdot s}^{-1}\right) \\
	&\gamma && \text { Relaxation factor }(-) \\
\end{align*}

\paragraph{Chapter 4}
\begin{align*}
	&C && \text { Radial clearance (m) } \\
	&d && \text { Body material damping }\left(\mathrm{N} \cdot \mathrm{s} \mathrm{~m}^{-1}\right) \\
	&E && \text { Elastic modulus (Pa) } \\
	&E_r && \text { Equivalent (reduced) elastic modulus (Pa) } \\
	&F_d && \text { Damping force (N) } \\
	&f_{\text {damp }} && \text { Damping factor }(-) \\
	&f_F && \text { Force vector (N) } \\
	&f_M && \text { Moment vector }(\mathrm{N} \cdot \mathrm{~m}) \\
	&f && \text { Force on partial mass (N) } \\
	&f^a && \text { External loads (N) } \\
	&f^* && \text { Non-linear excitation force }(\mathrm{N}) \\
	&G^* && \text { Dimensionless equivalent geometry }(-) \\
	&h_c && \text { Central film thickness }(\mathrm{m}) \\
	&I_C && \text { Inertia tensor of partial mass }\left(\mathrm{kg} \cdot \mathrm{~m}^2\right) \\
	&K && \text { Body stiffness matrix }\left(\mathrm{N} \cdot \mathrm{~m}^{-1}\right) \\
	&k && \text { Body material stiffness }\left(\mathrm{N} \cdot \mathrm{~m}^{-1}\right) \\
	&K_c && \text { Contact stiffness }\left(\mathrm{N} \cdot \mathrm{~m}^{-1}\right) \\
	&K_b && \text { Total bearing stiffness }\left(\mathrm{N} \cdot \mathrm{~m}^{-1}\right) \\
	&K_{E H L} && \text { EHL film stiffness }\left(\mathrm{N} \cdot \mathrm{~m}^{-1}\right) \\
	&l_a && \text { Active length of roller }(\mathrm{m}) \\
	&l && \text { Length of roller slice }(\mathrm{m}) \\
	&m && \text { Mass of partial mass }(\mathrm{kg}) \\
	&M && \text { Mass matrix of body }(\mathrm{kg}) \\
	&N && \text { Partial mass number }(-) \\
	&n && \text { Degree of freedom }(-) \\
	&p^* && \text { Non-linear inertia terms }\left(\mathrm{kg} \cdot \mathrm{~m}^2\right) \\
	&q && \text { Displacement }(\mathrm{m}) \\
	&\ddot{q} && \text { Velocity }\left(\mathrm{m} \cdot \mathrm{~s}^{-1}\right) \\
	&\ddot{q} && \text { Acceleration }\left(\mathrm{m} \cdot \mathrm{~s}^{-2}\right) \\
	&R && \text { Bearing inner race radius }(\mathrm{m}) \\
	&r && \text { Roller radius }(\mathrm{m}) \\
	&R_r && \text { Equivalent radius of contact }(\mathrm{m}) \\
\end{align*}

\pagebreak

\begin{align*}
	&s && \text { Slice number }(-) \\
	&T && \text { Total contact moment }(\mathrm{N} \cdot \mathrm{~m}) \\
	&u_t && \text { Translational displacement of partial mass }(\mathrm{m}) \\
	&U^* && \text { Dimensionless speed parameter }(-) \\
	&W && \text { Total contact load (N) } \\
	&w && \text { Force per unit length }\left(\mathrm{N} \cdot \mathrm{~m}^{-1}\right) \\
	&x && \text { Displacement in x-direction }(\mathrm{m}) \\
	&x_c && \text { Conjunction } \mathrm{x} \text {-coordinate }(-) \\
	&y && \text { Displacement in y-direction }(\mathrm{m}) \\
	&y_c && \text { Conjunction y-coordinate }(-) \\
	&z && \text { Displacement in z-direction }(\mathrm{m}) \\
\end{align*}

\paragraph{Greek Symbols}
\begin{align*}
	&\theta && \text { Roller angular displacement }(\mathrm{rad}) \\
	&\emptyset && \text { Rotational displacement of partial mass (rad) } \\
	&\alpha && \text { Pressure viscosity coefficient }\left(GPa^{-1}\right) \\
	&\delta && \text { Contact deformation }(\mathrm{m}) \\
	&\delta_m && \text { Material deformation }(\mathrm{m}) \\
	&\eta_0 && \text { Atmospheric lubricant dynamic viscosity (Pa } \cdot \mathrm{s}) \\
	&\rho_0 && \text { Lubricant inlet density }\left(\mathrm{kg} \cdot \mathrm{~m}^{-3}\right) \\
	&\omega && \text { Angular velocity of shaft (rad) } \\
\end{align*}

\paragraph{Chapter 5}
\begin{align*}
	&b && \text { Bias } \\
	&d && \text { Test field distance } \\
	&f && \text { Activation function } \\
	&h && \text { Hidden layer number } \\
	&N && \text { Number of neurons per hidden layer } \\
	&N_p && \text { Number of data points } \\
	&n_f && \text { Number of factors } \\
	&n_s && \text { Number of simulations } \\
	&R^2 && \text { Coefficient of determination } \\
	&t && \text { Total number of hidden layers } \\
	&t_i && \text { Target value } \\
	&u_l && \text { Lower normalised unit value } \\
	&u_n && \text { Upper normalised unit value } \\
	&W && \text { Weighting } \\
	&X && \text { Input value to neuron } \\
	&x && \text { Dimensional input value } \\
	&x_{i,j,LHS} && \text { Latin hypercube sampling element } \\
	&x_{max} && \text { Maximum dimensional value in dataset } \\
	&x_{min} && \text { Minimum dimensional value in dataset } \\
	&x_w && \text { Weighted sum of inputs plus bias } \\
	&\tilde{x} && \text { Normalised input or output parameter } \\
	&y_i && \text { Predicted value } \\
	&\overline{y} && \text { Mean of target sample } \\
	&Z_r && \text { Random number } \\
	&z && \text { Activated output of a neuron } \\
\end{align*}

\paragraph{Greek Symbols}
\begin{align*}
	&\xi && \text { MaxiMin application factor } \\
	&\gamma^{\prime} && \text { Regularisation adjustment factor } \\
\end{align*}


%\end{abstract}
