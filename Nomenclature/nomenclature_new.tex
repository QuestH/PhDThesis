% ************************** Thesis Abstract *****************************
% Use `abstract' as an option in the document class to print only the titlepage and the abstract.
%\begin{abstract}
\addcontentsline{toc}{chapter}{Nomenclature}

\chapter*{Nomenclature}

The nomenclature of all symbols used in the body of this work is given in the following sections. This work combines concepts from various research domains, including multi-body dynamics and elastohydrodynamic lubrication. Since each field has its own established notation, overlap for certain symbol invariably occurs. The nomenclature is therefore divided by chapter to maintain convention and reduce potential confusion.

\begin{align*}
	&A &&\text{Jacobian matrix AND Cross-sectional area}\\
	&A^{\dagger} &&\text{Pseudo-inverse Jacobian matrix}\\
	&B &&\text{Susceptance}\\
	&\mathbf{B} &&\text{Magnetic Flux}\\
	&C &&\text{Capacitance}\\
	&C_m &&\text{Measured capacitance}\\
	&c &&\text{Specific capacitance}\\
	&c^* &&\text{Complex permittivity}\\
	&\mathbf{D} &&\text{Electric displacement}\\
	&\mathbf{E} &&\text{Electric field}\\
	&E_l &&\text{The area in contact with electrode } l \text{ on } \partial \Omega \\	
	&\mathbf{H} &&\text{Magnetic field strength}\\
	&f_{nj} &&\text{Weighting value for tissue at } j \text{th element at FEM node } n \\
	&G &&\text{Conductance}\\
	&G_m &&\text{Measured conductance}\\
	&H^1 &&\text{A Hilbert space}\\	
	&\mathbf{I} \text{\textcolor{white}{iii}} I &&\text{Current}\\	
	&\mathbf{I}_l &&\text{Current at electrode } l \\
	&i &&\text{The imaginary number AND Frequency index}\\	
	&\mathbf{J} &&\text{Electric current density}\\	
	&\mathbf{J}_c &&\text{Conduction current density}\\
	&\mathbf{J}_s &&\text{Source current density}\\	
	&j &&\text{Boundary current density AND Tissue index}\\
	&K &&\text{Magnitude of the wavenumber vector}\\
\end{align*}

\pagebreak
\begin{align*}
	&L &&\text{Total number of electrodes AND Lead inductance}\\
	&l &&\text{Electrode numeration AND Negative log-likelihood AND Length AND Electrode channel}\\
	&N &&\text{Number of current sources AND Number of measurements}\\
	&n &&\text{FEM node element index}\\
	&\mathbf{n} &&\text{The unit vector of the normal to surface } \partial \Omega \\
	&R &&\text{Resistance}\\
	&r &&\text{Correlation coefficient}\\
	&S &&\text{A surface}\\
	&T &&\text{Total number of tissues}\\
	&t &&\text{Time}\\
	&t_j &&\text{Tissue number at } j\\
	&V &&\text{Voltage}\\		
	&V_l &&\text{Voltage at electrode } l \\	
	&V_{f_i} &&\text{Voltage at frequency } i \\
	&V_{f_{ref}} &&\text{Voltage at the reference frequency}\\
	&v &&\text{A set of test functions}\\
	&X &&\text{Reactance}\\
	&\mathbf{Y} \text{\textcolor{white}{iii}} Y &&\text{Admittance}\\
	&Z &&\text{Impedance}\\
	&z &&\text{Impedivity}\\
	&z_l &&\text{Contact impedance between electrode } l \text{ and the tissue on } E_l \\
	&\beta &&\text{Susceptivity}\\
	&\Gamma &&\text{Total boundary of } \partial \Omega \text{ that is in contact with an electrode}\\
	&\Gamma ' &&\partial \Omega \text{ not in contact with an electrode}\\
	&\gamma && \text{Admittivity}\\
	&\gamma_l && \text{Boundary admittivity on } E_l \\
	&\delta &&\text{Loss angle}\\
	&\partial \Omega &&\text{The surface area of volume } \Omega \\
	&\epsilon &&\text{Electrical permittivity}\\
	&\epsilon_r &&\text{Relative permittivity}\\
\end{align*}

\pagebreak
\begin{align*}
	&\epsilon_0 &&\text{Permittivity of free space}\\
	&\epsilon' &&\text{Real permittivity}\\
	&\epsilon'' &&\text{Imaginary permittivity}\\
	&\epsilon_r'' &&\text{Out of phase loss factor}\\
	&\boldsymbol{\Lambda} &&\text{Forward transformation map}\\
	&\lambda &&\text{Characteristic spatial wavelength}\\
	&\mu &&\text{Magnetic permittivity}\\
	&\rho' &&\text{Real resistivity}\\
	&\rho'' &&\text{Imaginary resistivity}\\
	&\rho &&\text{Resistivity}\\
	&\sigma \text{\textcolor{white}{iii}} \sigma &&\text{Electrical Conductivity}\\
	&\sigma_{ij} &&\text{Conductivity of tissue } j \text{ at frequency } i \\
	&\sigma_{n} &&\text{Conductivity of tissue at FEM node } n \\
	&\sigma^{tj} &&\text{Conductivity of tissue at } j \text{th element}\\
	&\tau &&\text{Regularisation parameter}\\
	&\phi &&\text{Scalar electric potential}\\
	&\boldsymbol{\Psi} &&\text{Regularising function}\\
	&\Omega &&\text{A volume}\\
	&\omega &&\text{Angular frequency}\\
	&\omega_{i} &&\text{Angular frequency at } i \text{th element}\\
	&\Delta T &&\text{Temperature difference}\\
	&\Delta V &&\text{Voltage difference}\\
	&\Delta V_{FD_i} &&\text{Frequency difference voltage at frequency } i \text{ with respect to}\\
	&\text{\textcolor{white}{iii}} &&\text{the reference frequency}\\
	&\Delta V_{FDA_i} &&\text{Frequency difference voltage at frequency } i \text{ with respect to}\\
	&\text{\textcolor{white}{iii}} &&\text{frequency } i - 1 \\
	&\Delta V_{WFDA_i} &&\text{Weighted frequency difference adjacent voltage at frequency } i \\
	&\text{\textcolor{white}{iii}} &&\text{with respect to frequency } i - 1 \\
	&\Delta \sigma &&\text{Conductivity difference}\\			
	&\tan \delta &&\text{Dissipation factor}\\
\end{align*}

%\end{abstract}
