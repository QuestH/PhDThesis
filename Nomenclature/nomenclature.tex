% ************************** Thesis Abstract *****************************
% Use `abstract' as an option in the document class to print only the titlepage and the abstract.
%\begin{abstract}
\addcontentsline{toc}{chapter}{Nomenclature}

\chapter*{Nomenclature}
\begin{align*}
	&\text{EIT} &&\text{Electrical impedance Tomography}\\
	&\text{ECG} &&\text{Electrocardiogram}\\
	&\text{MFEIT} &&\text{Multifrequency Electrical Impedance Tomography}\\
	&\text{CT} &&\text{Computerised Tomography}\\
	&\text{MRI} &&\text{Magnetic Resonance Imaging}\\
	&\text{AC} &&\text{Alternating Current}\\
	&\text{CSF} &&\text{Cerebrospinal Fluid}\\
	&\text{APT} &&\text{Applied Potential Tomography}\\
	&\text{FEM} &&\text{Finite Element Method}\\
	&\nabla &&\text{The vector differential operator}\\
	&\boldsymbol{E} &&\text{Electric field}\\
	&i &&\text{The imaginary number AND Frequency numeration}\\
	&\omega &&\text{Angular frequency}\\
	&\mu &&\text{Magnetic permittivity}\\
	&\boldsymbol{H} &&\text{Magnetic field strength}\\
	&\boldsymbol{J} &&\text{Electric current density}\\
	&\epsilon &&\text{Electrical permittivity}\\
	&\boldsymbol{J}_c &&\text{Conduction current density}\\
	&\boldsymbol{J}_s &&\text{Source current density}\\
	&\boldsymbol{\sigma} \text{\textcolor{white}{iii}} \sigma &&\text{Electrical Conductivity}\\
	&\phi &&\text{Scalar electric potential}\\
	&\Omega &&\text{A volume}\\
	&\partial \Omega &&\text{The surface area of volume } \Omega \\
	&j &&\text{Boundary current density AND Tissue numeration}\\
\end{align*}

\pagebreak
\begin{align*}
	&\boldsymbol{n} &&\text{The unit vector of the normal to surface } \partial \Omega \\
	&\text{CEM} &&\text{Complete Electrode Model}\\
	&\boldsymbol{I} \text{\textcolor{white}{iii}} I &&\text{Current}\\
	&\boldsymbol{Y} \text{\textcolor{white}{iii}} Y &&\text{Admittance}\\
	&\boldsymbol{V} \text{\textcolor{white}{iii}} V &&\text{Voltage}\\
	&l &&\text{Electrode numeration AND Negative log-likelihood AND Length}\\
	&L &&\text{Total number of electrodes}\\
	&E_l &&\text{The area in contact with electrode } l \text{ on } \partial \Omega \\
	&\Gamma ' &&\partial \Omega \text{ not in contact with an electrode}\\
	&\gamma && \text{Admittivity}\\
	&\gamma_l && \text{Boundary admittivity on } E_l \\
	&\partial && \text{Partial derivitive}\\
	&dS &&\text{A surface derivitive}\\
	&\boldsymbol{I}_l &&\text{Current at electrode } l \\
	&\boldsymbol{V}_l \text{\textcolor{white}{iii}} V_l &&\text{Voltage at electrode } l \\
	&z_l &&\text{Contact impedance between electrode } l \text{ and the tissue on } E_l \\
	&v &&\text{A set of test functions}\\
	&H^1 &&\text{A Hilbert space}\\
	&\Delta \boldsymbol{V} &&\text{Voltage difference}\\
	&\Delta \boldsymbol{\sigma} &&\text{Conductivity difference}\\
	&A &&\text{Jacobian matrix}\\
	&A^{\dagger} &&\text{Pseudo-inverse Jacobian matrix}\\
	&\Delta \boldsymbol{V}_{FD_i} &&\text{Frequency difference voltage at frequency } i \text{ with respect to}\\
	&\text{\textcolor{white}{iii}} &&\text{the reference frequency}\\
	&\boldsymbol{V}_{f_i} &&\text{Voltage at frequency } i \\
	&\boldsymbol{V}_{f_{ref}} &&\text{Voltage at the reference frequency}\\
	&\text{FDA} &&\text{Frequency Difference Algorithm}\\
	&\Delta \boldsymbol{V}_{FDA_i} &&\text{Frequency difference voltage at frequency } i \text{ with respect to}\\
	&\text{\textcolor{white}{iii}} &&\text{frequency } i - 1 \\
	&\text{WFD} &&\text{Weighted Frequency Difference}\\
	&\alpha &&\text{Weighted frequency difference correction factor}\\
\end{align*}

\pagebreak
\begin{align*}
	&\text{WFDA} &&\text{Weighted Frequency Difference Adjacent}\\
	&\Delta \boldsymbol{V}_{WFDA_i} &&\text{Weighted frequency difference adjacent voltage at frequency } i \\
	&\text{\textcolor{white}{iii}} &&\text{with respect to frequency } i - 1 \\
	&T &&\text{Total number of tissues}\\
	&t_j &&\text{Tissue number } j \\
	&\sigma_{ij} &&\text{Conductivity of tissue } j \text{ at frequency } i \\
	&n &&\text{FEM node element numeration}\\
	&f_{nj} &&\text{Weighting value for tissue } j \text{ at element } n \\
	&\text{GREIT} &&\text{Graz Reconstruction Algorithm for Electrical Impedance Tomography}\\
	&\boldsymbol{\Lambda} &&\text{Forward transformation map}\\
	&\tau &&\text{Regularisation parameter}\\
	&\boldsymbol{\Psi} &&\text{Regularising function}\\
	&G &&\text{Conductance}\\
	&B &&\text{Susceptance}\\
	&C &&\text{Capacitance}\\
	&Z &&\text{Impedance}\\
	&R &&\text{Resistance}\\
	&X &&\text{Reactance}\\
	&A &&\text{Cross-sectional area}\\
	&\beta &&\text{Susceptivity}\\
	&z &&\text{Impedivity}\\
	&\rho' &&\text{Real resistivity}\\
	&\rho'' &&\text{Imaginary resistivity}\\
	&\rho &&\text{Resistivity}\\
	&c &&\text{Specific capacitance}\\
	&\epsilon_r &&\text{Relative permittivity}\\
	&\epsilon_0 &&\text{Permittivity of free space}\\
	&c^* &&\text{Complex permittivity}\\
	&\epsilon' &&\text{Real permittivity}\\
	&\epsilon'' &&\text{Imaginary permittivity}\\
	&\epsilon_r'' &&\text{Out of phase loss factor}\\
\end{align*}

\pagebreak
\begin{align*}
	&\delta &&\text{Loss angle}\\
	&\tan \delta &&\text{Dissipation factor}\\
	&\Delta T &&\text{Temperature difference}\\
	&\text{EEG} &&\text{Electroencephalogram}\\
	&r &&\text{Correlation coefficient}\\
	&\text{PTD} \text{\textcolor{white}{iii}} PTD &&\text{Pecentage Thickness Dipol\"e}\\
	&\text{HCT} &&\text{Haematocrit}\\
	&K &&\text{Magnitude of the wavenumber vector}\\
	&\text{LHS} &&\text{Left Hand Side}\\
	&\lambda &&\text{Characteristic spatial wavelength}\\
\end{align*}

%\end{abstract}
